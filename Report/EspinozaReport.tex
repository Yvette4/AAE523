\documentclass[conference]{IEEEtran}
\IEEEoverridecommandlockouts
% The preceding line is only needed to identify funding in the first footnote. If that is unneeded, please comment it out.
%\usepackage{cite}
\usepackage{amsmath,amssymb,amsfonts}
\usepackage{algorithmic}
\usepackage{graphicx}
\usepackage{textcomp}
\usepackage{xcolor}
%\def\BibTeX{{\rm B\kern-.05em{\sc i\kern-.025em b}\kern-.08em
%		T\kern-.1667em\lower.7ex\hbox{E}\kern-.125emX}}

\usepackage{hyperref}
\usepackage{biblatex} 
\addbibresource{refs.bib}

\begin{document}
	
	\title{Characterizing Deprived Urban Areas Using Remote Sensing}

	\author{\IEEEauthorblockN{1\textsuperscript{st} Yvette Espinoza}
		\IEEEauthorblockA{\textit{Software Engineer} \\
			\textit{Northrop Grumman}\\
			Redondo Beach, CA, USA \\
			yespinoz@purdue.edu}
	}

	\maketitle

% ------------------------------------------------------------------------------------------------------------
% ~ ~ ~ ~ ~ ~ ~ ~ ~ ~ ~ ~ ~ ~ ~ ~ ~ ~ ~ ~ ~ ~ ~ ~ ~ ~ ~ ~ ~ ~ ~ ~ ~ ~ ~ ~ ~ ~ ~ ~ ~ ~ ~ ~ ~ ~ ~ ~ ~ ~ ~ ~ ~ ~ 
% ------------------------------------------------------------------------------------------------------------
	\begin{abstract}
		The urban population has increased dramatically in the last few decades, but the rapid rate of urbanization has caused a strain on the available resources, leaving many to live in deprived, or impoverished areas.
		To address these socio-demographic issues policymakers typically rely on traditional survey-based data, like the census, but such data is complex to acquire and can quickly become outdated. Earth observations are the proposed solution to the gaps left by traditional data.
		Artificial intelligence and deep learning algorithms are being used to detect changes on the earth's surface, such as detecting new urban areas.
		New research has focused on classifying elements of the city itself, monitoring waste disposal sites and traffic to have a better understanding of the deprived areas and their needs.
		This project will analyze urban characteristics, what makes an area ‘deprived’, and discuss the uses of remote sensing in socio-demographic applications while attempting to use real data to extract the characteristics of a city.

	\end{abstract}

% ------------------------------------------------------------------------------------------------------------
% ~ ~ ~ ~ ~ ~ ~ ~ ~ ~ ~ ~ ~ ~ ~ ~ ~ ~ ~ ~ ~ ~ ~ ~ ~ ~ ~ ~ ~ ~ ~ ~ ~ ~ ~ ~ ~ ~ ~ ~ ~ ~ ~ ~ ~ ~ ~ ~ ~ ~ ~ ~ ~ ~ 
% ------------------------------------------------------------------------------------------------------------
	\section{Literature Review}
		It is estimated that half of the worlds population is currently living in cities, with that number expected to rise to 60\% by 2030, but the rapid rate of urbanization has not resulted in an equal increase in public amenities or affordable housing \cite{UN_Sustanability_Report}.
		The lack of affordable housing has led to the creation of informal settlements, commonly referred to as 'slums', or deprived urban areas (DUA).
		These areas of concentrated poverty often lack public amenities such as access to clean water or waste disposal, and are known to have hazardous effects on the inhabitants health \cite{Georgano_Stefanos_2021}.
		
		Poverty reduction, listed in the United Nation's Sustainable Development Goals, is a priority worldwide, though policies enacted and definition of poverty itself vary between countries.
		Some common definitions include the current financial status of residents, or the value of a household's assets, but such methods alone do not paint a clear picture of the situation, since low income households do not necessarily belong to a deprived area \cite{Merodio_Paloma_2021}. 
		Since research has found poverty is related to population growth, one reduction approach focuses on addressing deprived urban areas, thus improving living conditions and reducing the risk of natural hazards \cite{Lin_Li_2021}.
		
		The traditional methods for mapping deprived areas rely on field surveys, such as the census, however these methods are costly and have a long period between collections.
		Since settlements are constantly changing, the data quickly becomes outdated and unable to provide feedback on whether any policies are effective \cite{Williams_Trecia_2020}.
		The COVID-19 pandemic brought more attention to this problem when many field surveys were delayed during a time when data was needed to divert resources to the most affected communities.
		One solution to fill the data gaps left by the traditional methods is to use remote sensing.
		
	\subsection{Remote Sensing}
		There are many ways in which remote sensing can be used to get a better understanding of an area.
		At a high level satellite images can be used to train deep learning models that classify areas as urban or rural \cite{Guo_Jinxin_2019}, but research has also focused on smaller features, such as waste piles and vehicles.
		
		Radar systems are desirable for data collection because they are not affected by weather conditions such as clouds or lighting.
		The X band of Synthetic Aperture Radars (SAR), which has frequencies in the range of 8-12 GHz, produces high resolution images that can be analyzed for feature extraction \cite{Wurm_2017}.
		The disadvantage to SAR is the lack of large datasets that can be used for training the models that extract features from urban areas \cite{Shi_Wenzhong_2020}.
		% YVETTE -  maybe add more on SAR here...
		
		Despite the limitations that optical systems have to weather conditions, they are the more popular data collection method for mapping deprived urban areas.
		Most of the urban characteristics analyzed are based on physical appearance which do not require the more detailed information SAR provides. 


	\subsection{Change Based Detection Algorithm}
		% Talk about the article on change detection algorithms, which are very important for socio-demographic applications --> eventually expand on what the project will attempt to do..
		
	\subsection{Texture Analysis}
		% 
		Texture analysis, characterizing an image by its texture content, is commonly used in remote sensing because of its effectiveness in classification \cite{Huang_2014}.
		One form of texture analysis uses the gray level co-occurrence matrix (GLCM) which contains a mapping of how often a pixel with intensity $i$ appears alongside a pixel of intensity $j$.
		The GLCM can be used to extract 14 texture statistics, with the most commonly used shown in Table \ref{tab:glcm_formulas} where $P_{ij}$ is the $(i,j)$ entry in the matrix. 
		
		% https://rstudio-pubs-static.s3.amazonaws.com/536921_af2c31c083544a3a9588da9c86692636.html
		\begin{table}[htbp]
			\caption{Common GLCM Statistics}
			\centering
			\begin{tabular}{ll} % creating 2 columns left aligned
				\\[1ex]
				Statistic & Formula \\ % Entering row contents
				\hline\hline %inserting double-line
				Energy & $\displaystyle\sum_{i,j=0}^{N-1} P_{ij}^2$ \\
				Contrast & $\displaystyle\sum_{i,j=0}^{N-1} P_{ij}(i-j)^2$ \\
				Entropy & $\displaystyle\sum_{i,j=0}^{N-1} P_{ij}(-ln(P_{ij}))$ \\
				Variance & $\sigma_i = \displaystyle\sum_{i=0}^{N-1} P_{ij}(i-\mu_i)^2$ \\[1ex] % [1ex] adds vertical space
				\hline % inserts single-line
			\end{tabular}
			\label{tab:glcm_formulas}
		\end{table}
		
		Of the texture statistics the variance has the best performance when differentiating between DUA's and formal areas, since a high variance gives a sharp change in the pixels that usually denote building edges or a drastic change in the environment \cite{Kuffer_2015}.
		The other statistics also provide valuable information, with contrast used to map DUA expansions and entropy as a validation step, since DUA's naturally have high entropy \cite{Kuffer_2016}. 
		
		
% ------------------------------------------------------------------------------------------------------------
% ~ ~ ~ ~ ~ ~ ~ ~ ~ ~ ~ ~ ~ ~ ~ ~ ~ ~ ~ ~ ~ ~ ~ ~ ~ ~ ~ ~ ~ ~ ~ ~ ~ ~ ~ ~ ~ ~ ~ ~ ~ ~ ~ ~ ~ ~ ~ ~ ~ ~ ~ ~ ~ ~ 
% ------------------------------------------------------------------------------------------------------------
%	\section{Simulation Environment}
%
%
%
%	\subsection{Another sub section}


% ------------------------------------------------------------------------------------------------------------
% ~ ~ ~ ~ ~ ~ ~ ~ ~ ~ ~ ~ ~ ~ ~ ~ ~ ~ ~ ~ ~ ~ ~ ~ ~ ~ ~ ~ ~ ~ ~ ~ ~ ~ ~ ~ ~ ~ ~ ~ ~ ~ ~ ~ ~ ~ ~ ~ ~ ~ ~ ~ ~ ~ 
% ------------------------------------------------------------------------------------------------------------
%	\section{Results}
%
%
%	\section{Conclusion}


% ------------------------------------------------------------------------------------------------------------
% ~ ~ ~ ~ ~ ~ ~ ~ ~ ~ ~ ~ ~ ~ ~ ~ ~ ~ ~ ~ ~ ~ ~ ~ ~ ~ ~ ~ ~ ~ ~ ~ ~ ~ ~ ~ ~ ~ ~ ~ ~ ~ ~ ~ ~ ~ ~ ~ ~ ~ ~ ~ ~ ~ 
% ------------------------------------------------------------------------------------------------------------
% USEFUL LINKS: 
% Worldview-3 dataset (mentioned in a few papers):
%	https://earth.esa.int/eogateway/catalog/worldview-3-full-archive-and-tasking


	\nocite{*}
	\printbibliography
	
\end{document}
